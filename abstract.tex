% $Log: abstract.tex,v $
% Revision 1.1  93/05/14  14:56:25  starflt
% Initial revision
% 
% Revision 1.1  90/05/04  10:41:01  lwvanels
% Initial revision
% 
%
%% The text of your abstract and nothing else (other than comments) goes here.
%% It will be single-spaced and the rest of the text that is supposed to go on
%% the abstract page will be generated by the abstractpage environment.  This
%% file should be \input (not \include 'd) from cover.tex.

Digital fabrication aims to bring the programmability of the digital world into the physical world and has the potential to radically transform the way we make things.  We are are interested in an emerging form of digital fabrication, where a small basis set of discrete parts types, called ``digital materials'', are reversible assembled into large assemblies with embedded functionality.  Objects constructed this way may be programmed with exotic functional behavior based on the composition of their constituent parts.\\

In this thesis I build an end-to end computer-aided design (CAD), simulation, and manufacturing (CAM) pipeline for digital materials that respects the discretization of the parts in its underlying software representation.  I propagate the same abstract geometric ``cell'' representation of parts from the design workflow into simulation.  I develop a dynamic model for simulating anisotropic, multimaterial assemblies of cells with embedded mechanical and electronic functionality based on local interactions.  I demonstrate that this model is equivalent to a shear-dominated Timoshenko Beam Element with non-linear treatment of angular displacements - allowing for large angular deformations to be simulated without costly remeshing.  I implement this simulation model and demonstrate its potential for parallelization by calculating each cell-cell interaction in a separate core of the GPU.  I compare my simulation results with a professional multiphysics software package.  I demonstrate that my tool facilitates rapid exploration of the design space around functional digital materials with several examples.
