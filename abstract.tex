% $Log: abstract.tex,v $
% Revision 1.1  93/05/14  14:56:25  starflt
% Initial revision
% 
% Revision 1.1  90/05/04  10:41:01  lwvanels
% Initial revision
% 
%
%% The text of your abstract and nothing else (other than comments) goes here.
%% It will be single-spaced and the rest of the text that is supposed to go on
%% the abstract page will be generated by the abstractpage environment.  This
%% file should be \input (not \include 'd) from cover.tex.
Digital fabrication aims to bring the programmability of the digital world into the physical world and has the potential to radically transform the way we make things.  Digital manufacturing processes deposit material bit by bit to build arbitrary 3D geometry.  By pattering multiple materials throughout an object with a specific spatial arrangement, programmable functional behavior is possible.  As the complexity of an object increases, the interactions between heterogenous parts becomes increasingly difficult for a human designer to intuit.  Rapid loops between design and simulation help to ground a design in its physical realities and make encoding functional behavior considerably less arduous.\\
\\
I'm developing a software workflow for designing large, hierarchical assemblies of multimaterial parts and simulating their physical behavior.  Currently, my system is concerned with three main application spaces: meter-kilometer scale aerospace structures, micro-scale electronics, and nano-scale self assembling DNA bricks.  The key difference between my software and other existing multi-physics simulation tools is the assumption that the elements of an assembly are arranged on a regular lattice.  This assumption simplifies the modeling of the system, allowing for much larger assemblies of parts to be simulated in a computationally efficient manner.  I believe that this approach will allow a human designer to explore design parameters more freely and open up the possibility of increased design automation through multiobjective optimization.