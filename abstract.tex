% $Log: abstract.tex,v $
% Revision 1.1  93/05/14  14:56:25  starflt
% Initial revision
% 
% Revision 1.1  90/05/04  10:41:01  lwvanels
% Initial revision
% 
%
%% The text of your abstract and nothing else (other than comments) goes here.
%% It will be single-spaced and the rest of the text that is supposed to go on
%% the abstract page will be generated by the abstractpage environment.  This
%% file should be \input (not \include 'd) from cover.tex.

l;afskjdflksdaj

%Digital fabrication aims to bring the programmability of the digital world into the physical world and has the potential to radically transform the way we make things.  Digital manufacturing processes deposit material bit by bit to build arbitrary 3D geometry.  A novel method of digital fabrication called "digital assembly" employs tiny robotic assemblers to pattern multiple materials with $\mu$M to nm resolution throughout an object.  Objects constructed this way may be programmed with exotic functional behavior based on the composition of their parts.
%\\
%
%I propose to build a CAD/simulation environment to explore the rich design space surrounding the digital assembly of electromechanical objects.  The bulk of this work centers around building an abstracted simulation engine for modeling electronic and mechanical properties of assemblies of parts and implementing a graphical CAD/simulation GUI around this engine.  Further work involves design studies of functional machines that could be constructed in this new design space.  My primary interest in this project is to design machine components essential to the construction of an assembler which is made from its own feedstock - capable of self replication. 

%This work draws on previous investigations by Von Neumann and others in the Cellular Automata (CA) community.  In CA literature, a machine that can be programmed to build any number of structures, including itself, is called a "universal constructor".  The basic requirements of a universal constructor are well formulated\cite{Neumann1966}, and many implementations exist in various cellular automata worlds (\href{https://www.youtube.com/watch?v=A8B5MbHPlH0}{Conway}, \href{https://en.wikipedia.org/wiki/Von_Neumann_universal_constructor#/media/File:Nobili_Pesavento_2reps.png}{Von Neuman}, \href{https://www.youtube.com/watch?v=PBXO_6Jn1fs}{CBlocks3D}).  However, these implementations violate basic physical laws - matter is created or destroyed at will, cells have no mass, interactions between cells are not fully modeled, and imaginary materials are required.  Through my thesis, I will begin to answer the question of how we will start engineering a universal constructor from real parts.