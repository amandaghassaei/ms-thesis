%% This is an example first chapter.  You should put chapter/appendix that you
%% write into a separate file, and add a line \include{yourfilename} to
%% main.tex, where `yourfilename.tex' is the name of the chapter/appendix file.
%% You can process specific files by typing their names in at the 
%% \files=
%% prompt when you run the file main.tex through LaTeX.

\singlespacing{

\chapter{Introduction}

An aspirational goal of digital fabrication is the bottom up programmable assembly of meter-scale objects with nanometer-scale precision.  With this technology, we could design materials with exotic physical properties and radically transform the way we make almost anything.  Current efforts aim to improve the precision and speed of top-down nanofabrication processes to make increasingly more complex nanodevices.  We believe a more scalable approach leverages the parallel actions of millions or billions of nanoscale assemblers, rather than a single, monolithic machine.  Though it sounds like science fiction, biology has demonstrated that this is possible; data encoded in DNA can be executed like a computer program to construct an immense assortment of molecular-scale machines, which together, coordinate the higher-level structures and functions of an organism.%yet proteins display a wide variety of functions and morphologies
\\
%approach to fine-grained control over matter is through the

In our proposed assembly system everything is constructed from a relatively small basis set of discrete feedstock, called ``digital materials''.  Drawing inspiration from the amino acid building blocks of biology, digital materials are reversibly joined in a highly parallel, serial assembly process to produce diverse, functional structures. Since construction takes place one nano-brick at a time, many assemblers work in parallel to build structures of any significant size.  Assemblers are designed so that they can be constructed from their own feedstock; assemblers build more assemblers and the rate of assembly scales exponentially.\\% Similarly, the protein machinery of biology is made primarily from the same basis set of about 20 reversibly-assembled amino acids.  

Parallel nano-assembly will look very different from the way we make things today, and opposes many of the assumptions baked into traditional Computer Aided Design (CAD),  Computer Aided Manufacturing (CAM), and robotics design.  Mechanical systems are built with discrete modules of rigid and flexural components assembled on a regular lattice, and electronics and controls are distributed spatially across a machine.  Large structures appear to be ``living'' in the sense that their surface is teeming with nano-robots, detecting and correcting errors, shuttling material feedstock around, and performing distributed sensing and actuation functions.  Machines receive instructions from their environment to coordinate various tasks.  The environment is highly structured, allowing locomotion systems to position themselves globally by counting local movements across a lattice.  %This assembly strategy follows from an existing line of research called ``digital assembly'', where discrete parts are assembled on a regular, periodic lattice.  
\\

This thesis outlines work toward fully integrated, end-to-end software workflows for digital materials that will help us to realize the aspirational future I've just described.  Current work is targeted at micro, milli, and macro-scale applications, with an eye toward scaling down to nanoscale.  The main contributions include:\\

\textbf{CAD:} a CAD environment for discrete construction that lowers the barriers to designing large, multimaterial assemblies of digital materials.  Hierarchical, parametric controls streamline the process of designing large assemblies of parts.   Abstraction of lattice geometry from its part decomposition simplifies the CAD interface and provides a meaningful underlying representation going into simulation.  CAD workflow is detailed in Chapters \ref{chap:CAD} and \ref{chap:implementation}.\\

\textbf{Simulation:} a dynamic physics engine that leverages the discrete construction of digital materials to simulate their electronic and mechanical behavior with better performance and less user hassle than professional multi-physics software packages.  This simulation tool has been GPU-accelerated so that it performs a dynamic simulation of hundreds of parts in realtime, allowing users to rapidly iterate on their designs.  Simulation methods are detailed in Chapter \ref{chap:functionSim} and evaluated against existing methods and software in Chapter \ref{chap:evaluation}.\\

\textbf{CAM:} a CAM workflow for the current iteration of macro-scale assemblers in development at CBA.  An abstracted machine definition allows a wide diversity of assemblers to be configured through the CAM interface, with hooks for custom g-code generation and machine settings.  The CAM workflow is detailed in Chapter \ref{chap:CAD}.

\section{Motivations}
 
The primary motivation for this work is to provide a platform for rapidly exploring the design space around digital materials in a physically realistic way.  This work will inform future research trajectories at CBA and in the broader field of programmable materials, modular robotics, and digital fabrication.   

\subsubsection{Multiphysics for the Masses}

The discrete, regular lattice geometry of digital assemblies reduces the complexity of design and simulation workflows and increases the computational efficiency of simulating hundreds or thousands of parts at once.   A side effect of this reduced complexity is that software tools for digital materials can be designed to have a very low technical barrier to entry.  Leveraging this, the tools developed in this thesis aim to be a kind of ``Minecraft with physics'': a multiphysics sandbox that even a novice user can quickly understand and operate.

\subsubsection{Experiments in Self-Replicating Systems}

The CAD/simulation tool I've built follows from a lineage of virtual construction environments and simulators where early explorations in distributed machine design and self-replication were first performed.  Many of these virtual environments have since been opened up to a larger, global community of researchers and enthusiasts.  In the fullness of time, I hope that my software tools will grow into an accessible platform for exploring physical, self-assembling systems based on the foundations of engineering and materials science rather than biology.   


%Future work looks toward hierarchical simulation and computational design optimization to discover even larger regions of parameter space than are currently possible (Chapter \ref{chap:futureWork}).  



% This work is not intended to result in a manual outlining all the necessary components for self-replication based on current technology, but rather as an 

%\subsection{Micro and Nano Robotics}
%
%%Nano structures allow for physical phenomena that are normally confined to the nanoscale to be accessed at the macroscale.  Gecko paper
%%nano structures affect the bulk material properties of macro-scale objects - strength, stiffness, friction, porosity, filtering, hydrophobicity, self cleaning, adhesion, conductivity...
%%
%%-relative cost of biological nano products - wood, down
%%-use of microbes to produce chemicals with high efficiency - insullin etc
%%-
%%-cost of nano-fabrication facitily, vibration specs, build size of machines, resolution
%
%\subsection{Artificial Self-Replicating Systems}
%
%



%
%
%machines make machines



%ease of minecraft design + physical reality
