%% This is an example first chapter.  You should put chapter/appendix that you
%% write into a separate file, and add a line \include{yourfilename} to
%% main.tex, where `yourfilename.tex' is the name of the chapter/appendix file.
%% You can process specific files by typing their names in at the 
%% \files=
%% prompt when you run the file main.tex through LaTeX.

\singlespacing{


\chapter{Micro Actuators}

supramolecular chemistry


In the sections below, a variety of actuator types are asses based on their performance characteristics.  The scaling laws of each actuator type are derived from the governing physics.

\subsection{Performance Characteristics Definitions}

\textbf{Actuation stress} ($\sigma$) is a measure of the applied force per unit cross-sectional area.  \textbf{Maximum actuation stress} ($\sigma_{max}$) gives the max impulse stress in a stroke of maximum work output.\\

\textbf{Actuation strain} ($\epsilon$) is a measure of the extension/contraction of an actuator relative to its nominal length. \textbf{Maximum actuation strain} ($\epsilon_{max}$) gives the max strain in a stroke of maximum work output.  The \textbf{strain resolution} ($\epsilon_{min}$) is the minimum step in $\epsilon$ possible for a given actuator.\\

\textbf{Actuator density} ($\rho$) gives the ratio of the mass of an actuator to its nominal volume in units of kg/m\textsuperscript{3}.  In this analysis, actuator density is considered independent of the mass of external power supplies, controllers, and fixturing.\\

\textbf{Actuator modulus} ($E$) is the ratio small changes in $\sigma$ to small changes in $\epsilon$ ($d\sigma / d\epsilon$) for a given control signal.\\

\textbf{Maximum actuation frequency} ($f_{max}$) is the max frequency of cyclic operation of a given actuator.  $f_{max}$ may require an external resetting force and may depend on external factors, such as heat dissipation or energy consumption. $f_{max}$ may be so low for some actuators, they should be considered only as single stroke actuators, rather than for cyclic operation.\\

For actuators in sustainable cyclic operation, \textbf{volumetric power} ($p$) is a measure of the mechanical power output per unit volume.\\

\textbf{Efficiency} ($\nu$) is the ratio of mechanical work output to energy input for one complete cycle of operation.\\


\section{Electromagnetic}



\subsection{Solenoid}

\subsection{Voice Coil}

\subsection{Motors}

\subsection{Electro Permanent}

\section{Electrostatic}

%\subsection{Electrostatic Comb Drive}

\section{Piezo}

Piezoelectric materials strain in the presence of an applied electric field $E$.
piezoelectricity $E$
electrostriction $E^{2}$
ferroelectricity (retain at E = 0)

 \begin{equation}\label{eq:diffEq}
\sigma_{max} = E\epsilon_{max}
  \end{equation}
\\
In piezoelectrics, $\epsilon_{max}$ is limited by the material type.  Piezoelectric materials have a maximum tolerable electric field, above which, the material operates in a different regime.
Low strain piezoelectrics: Quartz (SiO\textsubscript{2}), Lithium Niobate (LiNbO\textsubscript{3}), Lithium Tantalate (LiTaO\textsubscript{3})
High strain piezoelectrics: alloys of Lead Zircoate Titanate (PbZr\textsubscript{x}Ti\textsubscript{1-x}O\textsubscript{3}, called PZT)


\section{Shape Memory Alloy}

\section{Magentostrictive}

\section{Thermal}

%\subsection{Wax Actuator}

\section{Pneumatic}

\section{Ultrasonic}

\section{Biological Actuators}


\section{Conclusions}

Further comparison of the performance metrics of mechanical actuators is given in Huber et al \cite{Street1997}.

 is given in Table \ref{tab:actuatorTypes}.

\renewcommand{\arraystretch}{1.5}
%http://tex.stackexchange.com/questions/98388/how-to-make-table-with-rotated-table-headers-in-latex
\begin{table}[h] \label{tab:actuatorTypes}
    \centering
    \caption{Micro-Actuation mechanisms comparison chart.}
\begin{tabular}{ll | *{7}{c} }
    \\
    \multicolumn{2}{c}{Name} 
        & \mcrot{1}{l}{60}{Cost} & \mcrot{1}{l}{60}{Density} & \mcrot{1}{l}{60}{Energy Density} & \mcrot{1}{l}{60}{Efficiency} & \mcrot{1}{l}{60}{Scaling} & \mcrot{1}{l}{60}{Strain} & \mcrot{1}{l}{60}{Etc}\\
    \midrule \midrule

    \multirow{4}{*}{\rotatebox{90}{\textbf{\small{\hspace{17pt}Electromagnetic}}}}
    & Solenoid&
        x & x & x & xxx & x & xxx & -\\
    & Voice-Coil  &
        xx & xxx & xxx & xx & xxx 
        & xx & -\\
    \midrule
    \multirow{4}{*}{\rotatebox{90}{\textbf{\small{\hspace{18pt}Electrostatic}}}}
    & Comb Drive
        & x & x & - & - & x 
        & - & x \\
        
      \midrule
     \multirow{4}{*}{\rotatebox{90}{\textbf{\small{\hspace{14pt}Piezo}}}}
    & PZT
        & x & x & - & - & x 
        & - & x\\
        
        \midrule
     \multirow{4}{*}{\rotatebox{90}{\textbf{\small{\hspace{16pt}Thermal}}}}
    & Paraffin Wax
        & x & x & - & - & x 
        & - & x\\
        
        
                \midrule
     \multirow{4}{*}{\rotatebox{90}{\textbf{\small{\hspace{16pt}Pnuematic}}}}
    & Air-Driven
        & x & x & - & - & x 
        & - & x\\
    & Fluid-Driven 
        & - & - & x & - & - 
        & - & -\\
        
        
    \bottomrule
\end{tabular}
\end{table}


}
